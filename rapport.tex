\documentclass[]{article}
\usepackage[utf8]{inputenc}
\usepackage[T1]{fontenc}
\usepackage[dvipsnames]{xcolor}
\usepackage{lmodern}
\usepackage{amssymb,amsmath}
\usepackage{textgreek}
\usepackage{ifxetex,ifluatex}
\usepackage{todonotes}
\usepackage{xifthen}

\usepackage[american]{babel}

\usepackage[backend=biber,style=apa]{biblatex}
\addbibresource{DreamReplay.bib}
\DeclareLanguageMapping{american}{american-apa}

\newcommand\bla[1]{\colorbox{Apricot}{** \ifthenelse{\isempty{#1}}{bla bla}{#1} **}}

% use upquote if available, for straight quotes in verbatim environments
\IfFileExists{upquote.sty}{\usepackage{upquote}}{}
\ifnum 0\ifxetex 1\fi\ifluatex 1\fi=0 % if pdftex
  \usepackage[utf8]{inputenc}
\else % if luatex or xelatex
  \ifxetex
    \usepackage{mathspec}
    \usepackage{xltxtra,xunicode}
  \else
    \usepackage{fontspec}
  \fi
  \defaultfontfeatures{Mapping=tex-text,Scale=MatchLowercase}
  \newcommand{\euro}{€}
\fi
% use microtype if available
\IfFileExists{microtype.sty}{\usepackage{microtype}}{}
\ifxetex
  \usepackage[setpagesize=false, % page size defined by xetex
              unicode=false, % unicode breaks when used with xetex
              xetex]{hyperref}
\else
  \usepackage[unicode=true]{hyperref}
\fi

\usepackage{csquotes}


\hypersetup{breaklinks=true,
            pdfauthor={Paul Ecoffet},
            pdftitle={Rapport de stage},
            colorlinks=true,
            citecolor=blue,
            urlcolor=blue,
            linkcolor=magenta,
            pdfborder={0 0 0}}
\urlstyle{same}  % don't use monospace font for urls
\setlength{\parindent}{0pt}
\setlength{\parskip}{6pt plus 2pt minus 1pt}
\setlength{\emergencystretch}{3em}  % prevent overfull lines
%\setcounter{secnumdepth}{0}

\title{Rapport de stage}
\author{Paul Ecoffet}
\date{15 avril 2016}


\begin{document}
\maketitle

\begin{abstract}
  \bla{Reinforcement learning, navigation, replay, cost of exploration}
\end{abstract}

{
\hypersetup{linkcolor=black}
\setcounter{tocdepth}{3}
%\tableofcontents
}
\section{Introduction}\label{introduction}

\section{Reinforcement learning}\label{reinforcement-learning}

Reinforcement Learning is a kind of learning mechanism where an agent tries to maximise a reward by doing some specific action in an environment \parencite{sutton_reinforcement_1998}. \bla{}

\section{Learning by replay}\label{learning-by-replay}

\subsection{In reinforcement learning}
\label{sub:In reinforcement learning}

Reinforcement learning methods like Q-learning or Sarsa converge slowly and need a lot of samples to be efficient. These algorithms, based on the TD(\textlambda), use the sample once to improve their solution, then discard it \parencite{adam_experience_2012}.

To improve the efficiency of reinforcement learning methods, experience replay can be use. \textcite{vanseijen_deeper_2015} offers a good analysis about learning by replay. The goal of replay is to use the maximum of information an experience offers. Each sample is used several times, improving the solution several times. Compared to TD(0), replay techniques give a better convergence to the optimal solution with the same number of experiences. Reinforcement learning methods using replay are more expensive than TD(\textlambda) both in memory and in computations, though the memory and computational power needed can be reduced a lot as \textcite{vanseijen_deeper_2015} shows.

\subsubsection{Planning and replay?}
\label{subs:Planning and replay}

RL algorithms with planning use a model of their environment to infer behaviour they haven't done yet. The model is commonly a transition table between the possible state in the environment according to the action done by the agent. For exemple, if the agent has to solve a maze and has access to a transition table, then it can look up the table and know that if he is in the case \((3, 5)\) and it does the action ``go to the east'', it will be in \((3, 6)\).

The model can be either directly accessible or inferred by the agent. When the model is inferred, the agent builds explicitly the transition table according to its experiences. Dyna is a class of RL algorithms where the model is inferred by the agent. There is two phase during an iteration of Dyna. The first phase is \bla{} and the second is the \emph{planning} phase. During the planning phase, the agent improves its value function estimation with simulations according to its model: The agent generate a sample according to some rules (a probability distribution) and use its model to compute the value of this sample. Then, it updates its solution according to this simulation \parencite{sutton_dyna-style_2012}. It is interesting to note that the distribution used has no influence on whether the solution will converge or not, but has an influence on the speed of convergence. Therefore, choosing a good distribution can greatly improve the performance of Dyna. \textcite{sutton_dyna-style_2012} shows that choosing only already explored samples can have a good impact on performances. Therefore, Dyna can be efficient with a selection of already experienced samples, the planning phase is then a \emph{replay} phase.

\textcite{vanseijen_deeper_2015} highlights even more the strong similarity between RL methods using replay and RL methods using planning. First of all, both of them can be considered as model-based algorithm. Indeed, planning methods using an inferred model use an integrated version of samples experienced over time. The sample isn't discarded once the solution is updated.
RL algorithms using experience replay \bla{considered as model-based algo}


In addition to that, \textcite{vanseijen_deeper_2015} shows a complete equivalence in the value function approximation at each iteration between a TD(0) inspired algorithm with replay and Linear Dyna, a model-based RL algorithm. Though Linear Dyna is a model based algorithm and thought as a ``looking ahead'' algorithm and the replay algorithm as a ``looking backward'' algorithm, they do the exact same calculation.


\begin{itemize}
\item
  Equivalence between replay and planning as show in \cite{vanseijen_deeper_2015}.
\item
  Can one use Linear dyna with prioritised sweeping as a learning by replay method?
\end{itemize}

\subsection{In vivo}
\label{sub:In vivo}

Replays of sequences have been observed in the rat after navigation tasks \parencite{wilson_reactivation_1994,skaggs_replay_1996,davidson_hippocampal_2009,gupta_hippocampal_2010}.

\begin{itemize}
\item Replay during sleep and during hesitation
\item Replay can be either forward or backward
\item Replay can also be action that has never be done? --> Planning somewhat?
\item Replay aren't necessary about the most recent event?
\item Is ``surprise'' a good criterion?
\end{itemize}

\cite{gupta_hippocampal_2010}

\section{Model navigation
learning}\label{model-navigation-learning}

\section{Place cells}\label{place-cells}
Place cells are high level integrative neurons in the hippocampus. They have a specific receptive field

\begin{itemize}
  \item Highly integrated
  \item Response depending of the position of the rat
  \item Involved in replay
\end{itemize}


\section{What are the sequences that are replayed? How are they
selected?}\label{what-are-the-sequences-that-are-replayed-how-are-they-selected}

\textcite{gupta_hippocampal_2010} shows that the sequences that are replayed are not necessarly the most recent experiences. It shows also that the sequences that are replayed can be either forward or backward. In forward replays, the sequence is replayed in the same order as the sequence the rat has done. In backward replays, the sequence is done in the reverse order. What is the explanation for the selection of the sequence which is replayed, and its order? No criteria are proposed in \textcite{gupta_hippocampal_2010}. Yet, we can make interesting hypothesis knowing that \textcite{sutton_dyna-style_2012} suggests that there are very good samples to replay in a Dyna algorithm, and those sample are the sample where the difference between the expected value of a state and its real value is the greatest. It is consistant with \textcite{gupta_hippocampal_2010} which states that the most common replays are the one that leads to the reward or comes from the reward.

\section{Limitations}\label{limitations}


\printbibliography{}

\end{document}

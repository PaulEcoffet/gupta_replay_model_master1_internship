\documentclass[]{report}
\usepackage[T1]{fontenc}
\usepackage{lmodern}
\usepackage{amssymb,amsmath}
\usepackage{ifxetex,ifluatex}
\usepackage{fixltx2e} % provides \textsubscript
% use upquote if available, for straight quotes in verbatim environments
\IfFileExists{upquote.sty}{\usepackage{upquote}}{}
\ifnum 0\ifxetex 1\fi\ifluatex 1\fi=0 % if pdftex
  \usepackage[utf8]{inputenc}
\else % if luatex or xelatex
  \ifxetex
    \usepackage{mathspec}
    \usepackage{xltxtra,xunicode}
  \else
    \usepackage{fontspec}
  \fi
  \defaultfontfeatures{Mapping=tex-text,Scale=MatchLowercase}
  \newcommand{\euro}{€}
\fi
% use microtype if available
\IfFileExists{microtype.sty}{\usepackage{microtype}}{}
\ifxetex
  \usepackage[setpagesize=false, % page size defined by xetex
              unicode=false, % unicode breaks when used with xetex
              xetex]{hyperref}
\else
  \usepackage[unicode=true]{hyperref}
\fi
\hypersetup{breaklinks=true,
            bookmarks=true,
            pdfauthor={Paul Ecoffet},
            pdftitle={Rapport de stage},
            colorlinks=true,
            citecolor=blue,
            urlcolor=blue,
            linkcolor=magenta,
            pdfborder={0 0 0}}
\urlstyle{same}  % don't use monospace font for urls
\setlength{\parindent}{0pt}
\setlength{\parskip}{6pt plus 2pt minus 1pt}
\setlength{\emergencystretch}{3em}  % prevent overfull lines
\setcounter{secnumdepth}{0}

\title{Rapport de stage}
\author{Paul Ecoffet}
\date{8 avril 2016}


\begin{document}
\maketitle

{
\hypersetup{linkcolor=black}
\setcounter{tocdepth}{3}
\tableofcontents
}
\section{Introduction}\label{introduction}

\section{Reinforcement learning}\label{reinforcement-learning}

Reinforcement Learning is a kind of learning mechanism where an agent tries to maximise a reward by doing some specific action in an environment. Reinforcement learning is thought to be

\section{Learning by replay}\label{learning-by-replay}

\subsection{In reinforcement learning}
\label{sub:In reinforcement learning}

So as to improve the efficiency of reinforcement learning methods, \smallcaps{Someone} proposed to replay . \cite{vanseijen_deeper_2015} offers a good analysis about learning by replay. The goal of replay is to use the maximum of information an experience offers. Compared to TD(0), replay techniques offers a better convergence to the optimal solution with the same number of episodes. Replay methods are more expensive than TD(0) both in memory and in computations, though the memory and computational power needed can be reduced a lot as \cite{vanseijen_deeper_2015} shows.

\subsection{In vivo}
\label{sub:In vivo}

\cite{gupta_hippocampal_2010}

\begin{itemize}
\itemsep1pt\parskip0pt\parsep0pt
\item
  Equivalence between replay and planning as show in \cite{vanseijen_deeper_2015}.
\item
  Can one use Linear dyna with prioritised sweeping as a learning by replay method?
\item

\end{itemize}

\section{Modeling navigation
learning}\label{modeling-navigation-learning}

\section{Place cells}\label{place-cells}
Place cells are high level integrative neurons in the hippocampus. They have a specific

\section{What are the sequences that are replayed? How are they
selected?}\label{what-are-the-sequences-that-are-replayed-how-are-they-selected}

\section{Limitations}\label{limitations}


\bibliographystyle{apalike}
\bibliography{DreamReplay}


\end{document}
